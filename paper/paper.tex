\documentclass[12pt,twocolumn]{article}

\usepackage{times}
\usepackage{latexsym}
\usepackage{amsmath}
\usepackage{multirow}
\usepackage{url} 

\usepackage{float}
\usepackage{lingmacros}
\usepackage{tree-dvips}
\usepackage{graphicx}
\usepackage{supertabular}
\usepackage{array}
\usepackage[english]{babel}

\usepackage{multicol}

\DeclareMathOperator*{\argmax}{arg\,max}

%%%%%%%%%%%%%%%%%%%%%%%%%%%%%%%%%%%%%%%%%% Title and Abstract %%%%%%%%%%%%%%%%%%%%%%%%%%%%%%%%%%%%%%%%%%



\begin{multicols}{1}
\title{Evolving Shepherding Behavior with a Genetic Programming Algorithm}
\author{
Josh Brule \\
Department of Computer Science \\
University of Maryland \\
{\tt jtcbrule@gmail.com} \\
\and
Kevin Engel \\
Department of Computer Science \\
University of Maryland \\
{\tt kevin.t.engel@gmail.com} \\
\and
Nick Fung \\
Department of Computer Science \\
University of Maryland \\
{\tt nfung13@gmail.com} \\
\and
Isaac Julien \\
Department of Computer Science \\
University of Maryland \\
{\tt ijulien6@gmail.com} \\
 }
\date{}

\begin{document}
\maketitle
\end{multicols}


\begin{abstract}

\end{abstract}

%%%%%%%%%%%%%%%%%%%%%%%%%%%%%%%%%%%%%%%%%% Introduction %%%%%%%%%%%%%%%%%%%%%%%%%%%%%%%%%%%%%%%%%%


\section{Introduction}



%%%%%%%%%%%%%%%%%%%%%%%%%%%%%%%%%%%%%%%%%% Related Work %%%%%%%%%%%%%%%%%%%%%%%%%%%%%%%%%%%%%%%%%%

\section{Related Work}

While flocking is a popular topic, algorithms for shepherding are less well-studied, and to the best of our knowledge
no existing work attempts to evolve such an algorithm with Genetic Programming. Existing approaches to sheparding
train a predictive model as in\cite{sumpter}, or employ predefined strategies which may be combined to acheive a
goal \cite{bennet}.

(Lien et. al, 2005) studies shepherding behavior in an environment with multiple shepherds cooperating to control a flock \cite{lien}.
Shepherds, which exert a repulsive force on the flock, must find \emph{steering points} to influence the direction of the flock as
desired. The steering points for the group of shepherds form either a line or an arc on a side of the flock, and each shepard
chooses a steering point to approach based on one of several proposed heuristics.

(Sumpter, et. al, 1998) presents a machine vision system that models the position and velocity of a flock of animals \cite{sumpter}.
A Point Distribution Mode is used to generate features based on input from a camera mounted on a "Robotic Sheepdog,"
and these features are then used to estimate a probability distribution of the movement of the flock over time, conditional on its
previous locations and velocities. This probability distribution is estimated using competitive learning in a neural network.
Finally, the robot can herd a flock of animals toward a goal by a maximum likelihood estimate of the robot's own path.

(Bennet and Trafankowski, 2012) provides an analysis of flocking and herding algorithms, and also introduces a
herding algorithm based on specific strategies inspired by real sheepdogs\cite{bennet}.
\cite{bennet} also considers using one of several flocking strategies for the animals being herded, and finds
that the success of different a herding algorithm is often dependent on the flocking behavior.

(Cowling and Gmeinwieser, 2010) uses a combined top-down and bottom-up approach to provide realistic sheep herding
in the context of a game. A finite state machine associated with each sheepdog represents possible herding strategies,
such as circling, and the state of the FSM is controlled at the top level by an AI "shepherd" \cite{cowling}.




%%%%%%%%%%%%%%%%%%%%%%%%%%%%%%%%%%%%%%%%%% Results %%%%%%%%%%%%%%%%%%%%%%%%%%%%%%%%%%%%%%%%%%

\section{Results}

%%%%%%%%%%%%%%%%%%%%%%%%%%%%%%%%%%%%%%%%%% Discussion %%%%%%%%%%%%%%%%%%%%%%%%%%%%%%%%%%%%%%%%%%

\section{Discussion}


%%%%%%%%%%%%%%%%%%%%%%%%%%%%%%%%%%%%%%%%%% Bibliography %%%%%%%%%%%%%%%%%%%%%%%%%%%%%%%%%%%%%%%%%%

\begin{thebibliography}{9}


\bibitem{bennet}
Brandon Bennett and Matthew Trafankowski.
\emph{A Comparative Investigation of Herding Algorithms}.
2012; In proceeding of: Proceedings of the Symposium on Understanding and Modelling Collective Phenomena (UMoCoP) at the Artificial Intelligence and Simulation of Behviour World Congress (AISB-12)

\bibitem{sumpter}
N. Sumpter, A.J. Rulpitt, R. Vaughan, R.D. Tillett, and R.D. Boylel.
\emph{Learning Models of Animal Behaviaur for a Robotic Sheepdog}
1998; IAPR Wotkrhop on Machine Vision Applications. Nov 17-t9. 19%. MAuhan. Chlha Japan

\bibitem{cowling}
Peter Cowling and Christian Gmeinwieser.
\emph{AI for Herding Sheep}
2010; Proceedings of the Sixth AAAI Conference on Artificial Intelligence and Interactive Digital Entertainment.


\bibitem{lien}
Jyh-Ming Lien, Samuel Rodriguez, Jean-Phillipe Malric, and Nancy M. Amato.
\emph{Shepherding Behaviors with Multiple Shepherds}.
2005. Proceedings of the 2005 IEEE International Conference on Robotics and Automation

\end{thebibliography}




\end{document}
